% Options for packages loaded elsewhere
\PassOptionsToPackage{unicode}{hyperref}
\PassOptionsToPackage{hyphens}{url}
\documentclass[
  spanish,
]{article}
\usepackage{xcolor}
\usepackage[margin=1in]{geometry}
\usepackage{amsmath,amssymb}
\setcounter{secnumdepth}{-\maxdimen} % remove section numbering
\usepackage{iftex}
\ifPDFTeX
  \usepackage[T1]{fontenc}
  \usepackage[utf8]{inputenc}
  \usepackage{textcomp} % provide euro and other symbols
\else % if luatex or xetex
  \usepackage{unicode-math} % this also loads fontspec
  \defaultfontfeatures{Scale=MatchLowercase}
  \defaultfontfeatures[\rmfamily]{Ligatures=TeX,Scale=1}
\fi
\usepackage{lmodern}
\ifPDFTeX\else
  % xetex/luatex font selection
\fi
% Use upquote if available, for straight quotes in verbatim environments
\IfFileExists{upquote.sty}{\usepackage{upquote}}{}
\IfFileExists{microtype.sty}{% use microtype if available
  \usepackage[]{microtype}
  \UseMicrotypeSet[protrusion]{basicmath} % disable protrusion for tt fonts
}{}
\makeatletter
\@ifundefined{KOMAClassName}{% if non-KOMA class
  \IfFileExists{parskip.sty}{%
    \usepackage{parskip}
  }{% else
    \setlength{\parindent}{0pt}
    \setlength{\parskip}{6pt plus 2pt minus 1pt}}
}{% if KOMA class
  \KOMAoptions{parskip=half}}
\makeatother
\usepackage{color}
\usepackage{fancyvrb}
\newcommand{\VerbBar}{|}
\newcommand{\VERB}{\Verb[commandchars=\\\{\}]}
\DefineVerbatimEnvironment{Highlighting}{Verbatim}{commandchars=\\\{\}}
% Add ',fontsize=\small' for more characters per line
\usepackage{framed}
\definecolor{shadecolor}{RGB}{248,248,248}
\newenvironment{Shaded}{\begin{snugshade}}{\end{snugshade}}
\newcommand{\AlertTok}[1]{\textcolor[rgb]{0.94,0.16,0.16}{#1}}
\newcommand{\AnnotationTok}[1]{\textcolor[rgb]{0.56,0.35,0.01}{\textbf{\textit{#1}}}}
\newcommand{\AttributeTok}[1]{\textcolor[rgb]{0.13,0.29,0.53}{#1}}
\newcommand{\BaseNTok}[1]{\textcolor[rgb]{0.00,0.00,0.81}{#1}}
\newcommand{\BuiltInTok}[1]{#1}
\newcommand{\CharTok}[1]{\textcolor[rgb]{0.31,0.60,0.02}{#1}}
\newcommand{\CommentTok}[1]{\textcolor[rgb]{0.56,0.35,0.01}{\textit{#1}}}
\newcommand{\CommentVarTok}[1]{\textcolor[rgb]{0.56,0.35,0.01}{\textbf{\textit{#1}}}}
\newcommand{\ConstantTok}[1]{\textcolor[rgb]{0.56,0.35,0.01}{#1}}
\newcommand{\ControlFlowTok}[1]{\textcolor[rgb]{0.13,0.29,0.53}{\textbf{#1}}}
\newcommand{\DataTypeTok}[1]{\textcolor[rgb]{0.13,0.29,0.53}{#1}}
\newcommand{\DecValTok}[1]{\textcolor[rgb]{0.00,0.00,0.81}{#1}}
\newcommand{\DocumentationTok}[1]{\textcolor[rgb]{0.56,0.35,0.01}{\textbf{\textit{#1}}}}
\newcommand{\ErrorTok}[1]{\textcolor[rgb]{0.64,0.00,0.00}{\textbf{#1}}}
\newcommand{\ExtensionTok}[1]{#1}
\newcommand{\FloatTok}[1]{\textcolor[rgb]{0.00,0.00,0.81}{#1}}
\newcommand{\FunctionTok}[1]{\textcolor[rgb]{0.13,0.29,0.53}{\textbf{#1}}}
\newcommand{\ImportTok}[1]{#1}
\newcommand{\InformationTok}[1]{\textcolor[rgb]{0.56,0.35,0.01}{\textbf{\textit{#1}}}}
\newcommand{\KeywordTok}[1]{\textcolor[rgb]{0.13,0.29,0.53}{\textbf{#1}}}
\newcommand{\NormalTok}[1]{#1}
\newcommand{\OperatorTok}[1]{\textcolor[rgb]{0.81,0.36,0.00}{\textbf{#1}}}
\newcommand{\OtherTok}[1]{\textcolor[rgb]{0.56,0.35,0.01}{#1}}
\newcommand{\PreprocessorTok}[1]{\textcolor[rgb]{0.56,0.35,0.01}{\textit{#1}}}
\newcommand{\RegionMarkerTok}[1]{#1}
\newcommand{\SpecialCharTok}[1]{\textcolor[rgb]{0.81,0.36,0.00}{\textbf{#1}}}
\newcommand{\SpecialStringTok}[1]{\textcolor[rgb]{0.31,0.60,0.02}{#1}}
\newcommand{\StringTok}[1]{\textcolor[rgb]{0.31,0.60,0.02}{#1}}
\newcommand{\VariableTok}[1]{\textcolor[rgb]{0.00,0.00,0.00}{#1}}
\newcommand{\VerbatimStringTok}[1]{\textcolor[rgb]{0.31,0.60,0.02}{#1}}
\newcommand{\WarningTok}[1]{\textcolor[rgb]{0.56,0.35,0.01}{\textbf{\textit{#1}}}}
\usepackage{graphicx}
\makeatletter
\newsavebox\pandoc@box
\newcommand*\pandocbounded[1]{% scales image to fit in text height/width
  \sbox\pandoc@box{#1}%
  \Gscale@div\@tempa{\textheight}{\dimexpr\ht\pandoc@box+\dp\pandoc@box\relax}%
  \Gscale@div\@tempb{\linewidth}{\wd\pandoc@box}%
  \ifdim\@tempb\p@<\@tempa\p@\let\@tempa\@tempb\fi% select the smaller of both
  \ifdim\@tempa\p@<\p@\scalebox{\@tempa}{\usebox\pandoc@box}%
  \else\usebox{\pandoc@box}%
  \fi%
}
% Set default figure placement to htbp
\def\fps@figure{htbp}
\makeatother
\ifLuaTeX
\usepackage[bidi=basic]{babel}
\else
\usepackage[bidi=default]{babel}
\fi
% get rid of language-specific shorthands (see #6817):
\let\LanguageShortHands\languageshorthands
\def\languageshorthands#1{}
\setlength{\emergencystretch}{3em} % prevent overfull lines
\providecommand{\tightlist}{%
  \setlength{\itemsep}{0pt}\setlength{\parskip}{0pt}}
\usepackage{booktabs}
\usepackage{longtable}
\usepackage{array}
\usepackage{multirow}
\usepackage{wrapfig}
\usepackage{float}
\usepackage{colortbl}
\usepackage{pdflscape}
\usepackage{tabu}
\usepackage{threeparttable}
\usepackage{threeparttablex}
\usepackage[normalem]{ulem}
\usepackage{makecell}
\usepackage{xcolor}
\usepackage{bookmark}
\IfFileExists{xurl.sty}{\usepackage{xurl}}{} % add URL line breaks if available
\urlstyle{same}
\hypersetup{
  pdftitle={Tarea 5},
  pdfauthor={Juan M Karawacki; Bruno Pintos},
  pdflang={es},
  hidelinks,
  pdfcreator={LaTeX via pandoc}}

\title{Tarea 5}
\author{Juan M Karawacki \and Bruno Pintos}
\date{9-17-2025}

\begin{document}
\maketitle

\subsection{Ejercicio 4.19 (Test de
Bechdel)}\label{ejercicio-4.19-test-de-bechdel}

En este ejercicio analizaremos \(\pi\), la proporción de películas que
pasan el test de Bechdel, usando los datos de \texttt{bechdel}. Para
cada escenario a continuación, especifica el modelo posterior de \(\pi\)
y calcula la media y la moda posteriores.

\subsubsection{1. John tiene una priori plana Beta(1, 1) y analiza
películas del año
1980.}\label{john-tiene-una-priori-plana-beta1-1-y-analiza-peluxedculas-del-auxf1o-1980.}

\pandocbounded{\includegraphics[keepaspectratio]{tarea_5_files/figure-latex/unnamed-chunk-3-1.pdf}}

Partimos de que la priori de John es una Beta(1, 1), es decir, una
distribución uniforme sobre el intervalo {[}0,1{]}. Esto implica que
inicialmente todos los valores posibles de \(\pi\) son igualmente
probables, por lo tanto la priori no aporta información real previa.

\begin{verbatim}
## Warning in attr(x, "align"): 'xfun::attr()' está en desuso.
## Utilizar 'xfun::attr2()' en su lugar.
## Ver help("Deprecated")
\end{verbatim}

\begin{verbatim}
## Warning in attr(.knitEnv$meta, "knit_meta_id"): 'xfun::attr()' está en desuso.
## Utilizar 'xfun::attr2()' en su lugar.
## Ver help("Deprecated")
## Warning in attr(.knitEnv$meta, "knit_meta_id"): 'xfun::attr()' está en desuso.
## Utilizar 'xfun::attr2()' en su lugar.
## Ver help("Deprecated")
## Warning in attr(.knitEnv$meta, "knit_meta_id"): 'xfun::attr()' está en desuso.
## Utilizar 'xfun::attr2()' en su lugar.
## Ver help("Deprecated")
## Warning in attr(.knitEnv$meta, "knit_meta_id"): 'xfun::attr()' está en desuso.
## Utilizar 'xfun::attr2()' en su lugar.
## Ver help("Deprecated")
## Warning in attr(.knitEnv$meta, "knit_meta_id"): 'xfun::attr()' está en desuso.
## Utilizar 'xfun::attr2()' en su lugar.
## Ver help("Deprecated")
## Warning in attr(.knitEnv$meta, "knit_meta_id"): 'xfun::attr()' está en desuso.
## Utilizar 'xfun::attr2()' en su lugar.
## Ver help("Deprecated")
## Warning in attr(.knitEnv$meta, "knit_meta_id"): 'xfun::attr()' está en desuso.
## Utilizar 'xfun::attr2()' en su lugar.
## Ver help("Deprecated")
## Warning in attr(.knitEnv$meta, "knit_meta_id"): 'xfun::attr()' está en desuso.
## Utilizar 'xfun::attr2()' en su lugar.
## Ver help("Deprecated")
## Warning in attr(.knitEnv$meta, "knit_meta_id"): 'xfun::attr()' está en desuso.
## Utilizar 'xfun::attr2()' en su lugar.
## Ver help("Deprecated")
## Warning in attr(.knitEnv$meta, "knit_meta_id"): 'xfun::attr()' está en desuso.
## Utilizar 'xfun::attr2()' en su lugar.
## Ver help("Deprecated")
## Warning in attr(.knitEnv$meta, "knit_meta_id"): 'xfun::attr()' está en desuso.
## Utilizar 'xfun::attr2()' en su lugar.
## Ver help("Deprecated")
## Warning in attr(.knitEnv$meta, "knit_meta_id"): 'xfun::attr()' está en desuso.
## Utilizar 'xfun::attr2()' en su lugar.
## Ver help("Deprecated")
## Warning in attr(.knitEnv$meta, "knit_meta_id"): 'xfun::attr()' está en desuso.
## Utilizar 'xfun::attr2()' en su lugar.
## Ver help("Deprecated")
## Warning in attr(.knitEnv$meta, "knit_meta_id"): 'xfun::attr()' está en desuso.
## Utilizar 'xfun::attr2()' en su lugar.
## Ver help("Deprecated")
\end{verbatim}

\begin{verbatim}
## Warning in attr(x, "align"): 'xfun::attr()' está en desuso.
## Utilizar 'xfun::attr2()' en su lugar.
## Ver help("Deprecated")
\end{verbatim}

\begin{verbatim}
## Warning in attr(x, "format"): 'xfun::attr()' está en desuso.
## Utilizar 'xfun::attr2()' en su lugar.
## Ver help("Deprecated")
\end{verbatim}

\begin{longtable}[t]{ccc}
\caption{\label{tab:unnamed-chunk-5}Películas 1980 – Test de Bechdel}\\
\toprule
Resultado & Cantidad & Proporción\\
\midrule
FAIL & 10 & 0.71\\
PASS & 4 & 0.29\\
Total & 14 & 1.00\\
\bottomrule
\end{longtable}

Al filtrar las películas del año 1980, observamos que de las 14
películas evaluadas, 4 pasaron el test de Bechdel y 10 no lo pasaron.

\newpage

\pandocbounded{\includegraphics[keepaspectratio]{tarea_5_files/figure-latex/unnamed-chunk-6-1.pdf}}

Al actualizar la priori con los datos, obtenemos la posterior y dado que
la priori no era informativa, la posterior coincide perfectamente con la
likelihood ya que adopta toda la información que recibe.

En este caso, la distribución posterior es:
\(\pi \sim \text{Beta}(5, 11)\), esto lo sabemos ya que, como aprendimos
en la Tarea 4, usando
\texttt{summarize\_beta\_binomial(alpha\ =\ 1,\ beta\ =\ 1,\ y\ =\ 4,\ n\ =\ 14)}
automaticamente obtenemos tanto el \(\alpha_{\text{post}}\) como
\(\beta_{\text{post}}\)

\begin{verbatim}
## Warning in attr(x, "align"): 'xfun::attr()' está en desuso.
## Utilizar 'xfun::attr2()' en su lugar.
## Ver help("Deprecated")
\end{verbatim}

\begin{verbatim}
## Warning in attr(.knitEnv$meta, "knit_meta_id"): 'xfun::attr()' está en desuso.
## Utilizar 'xfun::attr2()' en su lugar.
## Ver help("Deprecated")
## Warning in attr(.knitEnv$meta, "knit_meta_id"): 'xfun::attr()' está en desuso.
## Utilizar 'xfun::attr2()' en su lugar.
## Ver help("Deprecated")
## Warning in attr(.knitEnv$meta, "knit_meta_id"): 'xfun::attr()' está en desuso.
## Utilizar 'xfun::attr2()' en su lugar.
## Ver help("Deprecated")
## Warning in attr(.knitEnv$meta, "knit_meta_id"): 'xfun::attr()' está en desuso.
## Utilizar 'xfun::attr2()' en su lugar.
## Ver help("Deprecated")
## Warning in attr(.knitEnv$meta, "knit_meta_id"): 'xfun::attr()' está en desuso.
## Utilizar 'xfun::attr2()' en su lugar.
## Ver help("Deprecated")
## Warning in attr(.knitEnv$meta, "knit_meta_id"): 'xfun::attr()' está en desuso.
## Utilizar 'xfun::attr2()' en su lugar.
## Ver help("Deprecated")
## Warning in attr(.knitEnv$meta, "knit_meta_id"): 'xfun::attr()' está en desuso.
## Utilizar 'xfun::attr2()' en su lugar.
## Ver help("Deprecated")
## Warning in attr(.knitEnv$meta, "knit_meta_id"): 'xfun::attr()' está en desuso.
## Utilizar 'xfun::attr2()' en su lugar.
## Ver help("Deprecated")
## Warning in attr(.knitEnv$meta, "knit_meta_id"): 'xfun::attr()' está en desuso.
## Utilizar 'xfun::attr2()' en su lugar.
## Ver help("Deprecated")
## Warning in attr(.knitEnv$meta, "knit_meta_id"): 'xfun::attr()' está en desuso.
## Utilizar 'xfun::attr2()' en su lugar.
## Ver help("Deprecated")
## Warning in attr(.knitEnv$meta, "knit_meta_id"): 'xfun::attr()' está en desuso.
## Utilizar 'xfun::attr2()' en su lugar.
## Ver help("Deprecated")
## Warning in attr(.knitEnv$meta, "knit_meta_id"): 'xfun::attr()' está en desuso.
## Utilizar 'xfun::attr2()' en su lugar.
## Ver help("Deprecated")
## Warning in attr(.knitEnv$meta, "knit_meta_id"): 'xfun::attr()' está en desuso.
## Utilizar 'xfun::attr2()' en su lugar.
## Ver help("Deprecated")
## Warning in attr(.knitEnv$meta, "knit_meta_id"): 'xfun::attr()' está en desuso.
## Utilizar 'xfun::attr2()' en su lugar.
## Ver help("Deprecated")
\end{verbatim}

\begin{verbatim}
## Warning in attr(x, "align"): 'xfun::attr()' está en desuso.
## Utilizar 'xfun::attr2()' en su lugar.
## Ver help("Deprecated")
\end{verbatim}

\begin{verbatim}
## Warning in attr(x, "format"): 'xfun::attr()' está en desuso.
## Utilizar 'xfun::attr2()' en su lugar.
## Ver help("Deprecated")
\end{verbatim}

\begin{longtable}[t]{ccc}
\caption{\label{tab:unnamed-chunk-8}Resumen: media y moda de pi}\\
\toprule
Modelo & Media & Moda\\
\midrule
Priori & 0.500 & NA\\
Posterior & 0.312 & 0.286\\
\bottomrule
\end{longtable}

En conclusión, la posterior tiene una media de 0.31 y una moda de 0.29
aproximadamente. Esto sugiere que, según los datos de 1980, la
proporción de películas que pasaron el test de Bechdel fue cercana al
30\%, lo cual tiene lógica con lo mencionado en la Tabla 1.

\subsubsection{2. Al día siguiente, John analiza películas del año 1990,
construyendo su análisis a partir del realizado el día
anterior.}\label{al-duxeda-siguiente-john-analiza-peluxedculas-del-auxf1o-1990-construyendo-su-anuxe1lisis-a-partir-del-realizado-el-duxeda-anterior.}

Ahora filtramos nuevamente las peliculas pero esta vez analizamos las de
los 90.

\begin{verbatim}
## Warning in attr(x, "align"): 'xfun::attr()' está en desuso.
## Utilizar 'xfun::attr2()' en su lugar.
## Ver help("Deprecated")
\end{verbatim}

\begin{verbatim}
## Warning in attr(.knitEnv$meta, "knit_meta_id"): 'xfun::attr()' está en desuso.
## Utilizar 'xfun::attr2()' en su lugar.
## Ver help("Deprecated")
## Warning in attr(.knitEnv$meta, "knit_meta_id"): 'xfun::attr()' está en desuso.
## Utilizar 'xfun::attr2()' en su lugar.
## Ver help("Deprecated")
## Warning in attr(.knitEnv$meta, "knit_meta_id"): 'xfun::attr()' está en desuso.
## Utilizar 'xfun::attr2()' en su lugar.
## Ver help("Deprecated")
## Warning in attr(.knitEnv$meta, "knit_meta_id"): 'xfun::attr()' está en desuso.
## Utilizar 'xfun::attr2()' en su lugar.
## Ver help("Deprecated")
## Warning in attr(.knitEnv$meta, "knit_meta_id"): 'xfun::attr()' está en desuso.
## Utilizar 'xfun::attr2()' en su lugar.
## Ver help("Deprecated")
## Warning in attr(.knitEnv$meta, "knit_meta_id"): 'xfun::attr()' está en desuso.
## Utilizar 'xfun::attr2()' en su lugar.
## Ver help("Deprecated")
## Warning in attr(.knitEnv$meta, "knit_meta_id"): 'xfun::attr()' está en desuso.
## Utilizar 'xfun::attr2()' en su lugar.
## Ver help("Deprecated")
## Warning in attr(.knitEnv$meta, "knit_meta_id"): 'xfun::attr()' está en desuso.
## Utilizar 'xfun::attr2()' en su lugar.
## Ver help("Deprecated")
## Warning in attr(.knitEnv$meta, "knit_meta_id"): 'xfun::attr()' está en desuso.
## Utilizar 'xfun::attr2()' en su lugar.
## Ver help("Deprecated")
## Warning in attr(.knitEnv$meta, "knit_meta_id"): 'xfun::attr()' está en desuso.
## Utilizar 'xfun::attr2()' en su lugar.
## Ver help("Deprecated")
## Warning in attr(.knitEnv$meta, "knit_meta_id"): 'xfun::attr()' está en desuso.
## Utilizar 'xfun::attr2()' en su lugar.
## Ver help("Deprecated")
## Warning in attr(.knitEnv$meta, "knit_meta_id"): 'xfun::attr()' está en desuso.
## Utilizar 'xfun::attr2()' en su lugar.
## Ver help("Deprecated")
## Warning in attr(.knitEnv$meta, "knit_meta_id"): 'xfun::attr()' está en desuso.
## Utilizar 'xfun::attr2()' en su lugar.
## Ver help("Deprecated")
## Warning in attr(.knitEnv$meta, "knit_meta_id"): 'xfun::attr()' está en desuso.
## Utilizar 'xfun::attr2()' en su lugar.
## Ver help("Deprecated")
\end{verbatim}

\begin{verbatim}
## Warning in attr(x, "align"): 'xfun::attr()' está en desuso.
## Utilizar 'xfun::attr2()' en su lugar.
## Ver help("Deprecated")
\end{verbatim}

\begin{verbatim}
## Warning in attr(x, "format"): 'xfun::attr()' está en desuso.
## Utilizar 'xfun::attr2()' en su lugar.
## Ver help("Deprecated")
\end{verbatim}

\begin{longtable}[t]{ccc}
\caption{\label{tab:unnamed-chunk-11}Películas 1990 – Test de Bechdel}\\
\toprule
Resultado & Cantidad & Proporción\\
\midrule
FAIL & 9 & 0.6\\
PASS & 6 & 0.4\\
Total & 15 & 1.0\\
\bottomrule
\end{longtable}

Esta vez, el 40\% de las películas aprobaron el test de Bechdel, lo que
representa una evidencia un poco más optimista que la observada en 1980
y dado que John ya actualizó su priori con esos datos, ahora utiliza
como priori la distribución: \(\pi \sim \text{Beta}(5, 11)\)

\pandocbounded{\includegraphics[keepaspectratio]{tarea_5_files/figure-latex/unnamed-chunk-12-1.pdf}}

\begin{verbatim}
## Warning in attr(x, "align"): 'xfun::attr()' está en desuso.
## Utilizar 'xfun::attr2()' en su lugar.
## Ver help("Deprecated")
\end{verbatim}

\begin{verbatim}
## Warning in attr(.knitEnv$meta, "knit_meta_id"): 'xfun::attr()' está en desuso.
## Utilizar 'xfun::attr2()' en su lugar.
## Ver help("Deprecated")
## Warning in attr(.knitEnv$meta, "knit_meta_id"): 'xfun::attr()' está en desuso.
## Utilizar 'xfun::attr2()' en su lugar.
## Ver help("Deprecated")
## Warning in attr(.knitEnv$meta, "knit_meta_id"): 'xfun::attr()' está en desuso.
## Utilizar 'xfun::attr2()' en su lugar.
## Ver help("Deprecated")
## Warning in attr(.knitEnv$meta, "knit_meta_id"): 'xfun::attr()' está en desuso.
## Utilizar 'xfun::attr2()' en su lugar.
## Ver help("Deprecated")
## Warning in attr(.knitEnv$meta, "knit_meta_id"): 'xfun::attr()' está en desuso.
## Utilizar 'xfun::attr2()' en su lugar.
## Ver help("Deprecated")
## Warning in attr(.knitEnv$meta, "knit_meta_id"): 'xfun::attr()' está en desuso.
## Utilizar 'xfun::attr2()' en su lugar.
## Ver help("Deprecated")
## Warning in attr(.knitEnv$meta, "knit_meta_id"): 'xfun::attr()' está en desuso.
## Utilizar 'xfun::attr2()' en su lugar.
## Ver help("Deprecated")
## Warning in attr(.knitEnv$meta, "knit_meta_id"): 'xfun::attr()' está en desuso.
## Utilizar 'xfun::attr2()' en su lugar.
## Ver help("Deprecated")
## Warning in attr(.knitEnv$meta, "knit_meta_id"): 'xfun::attr()' está en desuso.
## Utilizar 'xfun::attr2()' en su lugar.
## Ver help("Deprecated")
## Warning in attr(.knitEnv$meta, "knit_meta_id"): 'xfun::attr()' está en desuso.
## Utilizar 'xfun::attr2()' en su lugar.
## Ver help("Deprecated")
## Warning in attr(.knitEnv$meta, "knit_meta_id"): 'xfun::attr()' está en desuso.
## Utilizar 'xfun::attr2()' en su lugar.
## Ver help("Deprecated")
## Warning in attr(.knitEnv$meta, "knit_meta_id"): 'xfun::attr()' está en desuso.
## Utilizar 'xfun::attr2()' en su lugar.
## Ver help("Deprecated")
## Warning in attr(.knitEnv$meta, "knit_meta_id"): 'xfun::attr()' está en desuso.
## Utilizar 'xfun::attr2()' en su lugar.
## Ver help("Deprecated")
## Warning in attr(.knitEnv$meta, "knit_meta_id"): 'xfun::attr()' está en desuso.
## Utilizar 'xfun::attr2()' en su lugar.
## Ver help("Deprecated")
\end{verbatim}

\begin{verbatim}
## Warning in attr(x, "align"): 'xfun::attr()' está en desuso.
## Utilizar 'xfun::attr2()' en su lugar.
## Ver help("Deprecated")
\end{verbatim}

\begin{verbatim}
## Warning in attr(x, "format"): 'xfun::attr()' está en desuso.
## Utilizar 'xfun::attr2()' en su lugar.
## Ver help("Deprecated")
\end{verbatim}

\begin{longtable}[t]{ccc}
\caption{\label{tab:unnamed-chunk-14}Resumen: media y moda de pi – Películas 1990}\\
\toprule
Modelo & Media & Moda\\
\midrule
Priori & 0.312 & 0.286\\
Posterior & 0.355 & 0.345\\
\bottomrule
\end{longtable}

La priori ahora aporta información relevante basada en los datos previos
de 1980, de manera que la posterior se posiciona entre la evidencia
anterior y los nuevos datos de 1990 con distribución
\(\pi \sim \text{Beta}(11, 20)\). Este ajuste conservador combina la
información previa con la nueva, reflejándose en un aumento de la media
de \(\pi\) de 0.31 a 0.35, mientras que la moda también se desplaza
ligeramente, mostrando cómo la priori influye en la actualización de
nuestra creencia sin exagerar el efecto de los nuevos datos.

\subsubsection{3. El tercer día, John analiza películas del año 2000,
nuevamente acumulando sobre sus análi1is de los dos días
previos.}\label{el-tercer-duxeda-john-analiza-peluxedculas-del-auxf1o-2000-nuevamente-acumulando-sobre-sus-anuxe1li1is-de-los-dos-duxedas-previos.}

Volvemos a filtrar esta vez para las peliculas del 2000

\begin{verbatim}
## Warning in attr(x, "align"): 'xfun::attr()' está en desuso.
## Utilizar 'xfun::attr2()' en su lugar.
## Ver help("Deprecated")
\end{verbatim}

\begin{verbatim}
## Warning in attr(.knitEnv$meta, "knit_meta_id"): 'xfun::attr()' está en desuso.
## Utilizar 'xfun::attr2()' en su lugar.
## Ver help("Deprecated")
## Warning in attr(.knitEnv$meta, "knit_meta_id"): 'xfun::attr()' está en desuso.
## Utilizar 'xfun::attr2()' en su lugar.
## Ver help("Deprecated")
## Warning in attr(.knitEnv$meta, "knit_meta_id"): 'xfun::attr()' está en desuso.
## Utilizar 'xfun::attr2()' en su lugar.
## Ver help("Deprecated")
## Warning in attr(.knitEnv$meta, "knit_meta_id"): 'xfun::attr()' está en desuso.
## Utilizar 'xfun::attr2()' en su lugar.
## Ver help("Deprecated")
## Warning in attr(.knitEnv$meta, "knit_meta_id"): 'xfun::attr()' está en desuso.
## Utilizar 'xfun::attr2()' en su lugar.
## Ver help("Deprecated")
## Warning in attr(.knitEnv$meta, "knit_meta_id"): 'xfun::attr()' está en desuso.
## Utilizar 'xfun::attr2()' en su lugar.
## Ver help("Deprecated")
## Warning in attr(.knitEnv$meta, "knit_meta_id"): 'xfun::attr()' está en desuso.
## Utilizar 'xfun::attr2()' en su lugar.
## Ver help("Deprecated")
## Warning in attr(.knitEnv$meta, "knit_meta_id"): 'xfun::attr()' está en desuso.
## Utilizar 'xfun::attr2()' en su lugar.
## Ver help("Deprecated")
## Warning in attr(.knitEnv$meta, "knit_meta_id"): 'xfun::attr()' está en desuso.
## Utilizar 'xfun::attr2()' en su lugar.
## Ver help("Deprecated")
## Warning in attr(.knitEnv$meta, "knit_meta_id"): 'xfun::attr()' está en desuso.
## Utilizar 'xfun::attr2()' en su lugar.
## Ver help("Deprecated")
## Warning in attr(.knitEnv$meta, "knit_meta_id"): 'xfun::attr()' está en desuso.
## Utilizar 'xfun::attr2()' en su lugar.
## Ver help("Deprecated")
## Warning in attr(.knitEnv$meta, "knit_meta_id"): 'xfun::attr()' está en desuso.
## Utilizar 'xfun::attr2()' en su lugar.
## Ver help("Deprecated")
## Warning in attr(.knitEnv$meta, "knit_meta_id"): 'xfun::attr()' está en desuso.
## Utilizar 'xfun::attr2()' en su lugar.
## Ver help("Deprecated")
## Warning in attr(.knitEnv$meta, "knit_meta_id"): 'xfun::attr()' está en desuso.
## Utilizar 'xfun::attr2()' en su lugar.
## Ver help("Deprecated")
\end{verbatim}

\begin{verbatim}
## Warning in attr(x, "align"): 'xfun::attr()' está en desuso.
## Utilizar 'xfun::attr2()' en su lugar.
## Ver help("Deprecated")
\end{verbatim}

\begin{verbatim}
## Warning in attr(x, "format"): 'xfun::attr()' está en desuso.
## Utilizar 'xfun::attr2()' en su lugar.
## Ver help("Deprecated")
\end{verbatim}

\begin{longtable}[t]{ccc}
\caption{\label{tab:unnamed-chunk-17}Películas 2000 – Test de Bechdel}\\
\toprule
Resultado & Cantidad & Proporción\\
\midrule
FAIL & 34 & 0.54\\
PASS & 29 & 0.46\\
Total & 63 & 1.00\\
\bottomrule
\end{longtable}

Ahora, el 46\% de las películas aprobaron el test de Bechdel, la
evidencia más positiva hasta el momento, lo cual tiene sentido, ya que,
dada la forma en que funciona el test, es esperable que en tiempos más
modernos los resultados sean mejores.

\pandocbounded{\includegraphics[keepaspectratio]{tarea_5_files/figure-latex/unnamed-chunk-18-1.pdf}}

\begin{verbatim}
## Warning in attr(x, "align"): 'xfun::attr()' está en desuso.
## Utilizar 'xfun::attr2()' en su lugar.
## Ver help("Deprecated")
\end{verbatim}

\begin{verbatim}
## Warning in attr(.knitEnv$meta, "knit_meta_id"): 'xfun::attr()' está en desuso.
## Utilizar 'xfun::attr2()' en su lugar.
## Ver help("Deprecated")
## Warning in attr(.knitEnv$meta, "knit_meta_id"): 'xfun::attr()' está en desuso.
## Utilizar 'xfun::attr2()' en su lugar.
## Ver help("Deprecated")
## Warning in attr(.knitEnv$meta, "knit_meta_id"): 'xfun::attr()' está en desuso.
## Utilizar 'xfun::attr2()' en su lugar.
## Ver help("Deprecated")
## Warning in attr(.knitEnv$meta, "knit_meta_id"): 'xfun::attr()' está en desuso.
## Utilizar 'xfun::attr2()' en su lugar.
## Ver help("Deprecated")
## Warning in attr(.knitEnv$meta, "knit_meta_id"): 'xfun::attr()' está en desuso.
## Utilizar 'xfun::attr2()' en su lugar.
## Ver help("Deprecated")
## Warning in attr(.knitEnv$meta, "knit_meta_id"): 'xfun::attr()' está en desuso.
## Utilizar 'xfun::attr2()' en su lugar.
## Ver help("Deprecated")
## Warning in attr(.knitEnv$meta, "knit_meta_id"): 'xfun::attr()' está en desuso.
## Utilizar 'xfun::attr2()' en su lugar.
## Ver help("Deprecated")
## Warning in attr(.knitEnv$meta, "knit_meta_id"): 'xfun::attr()' está en desuso.
## Utilizar 'xfun::attr2()' en su lugar.
## Ver help("Deprecated")
## Warning in attr(.knitEnv$meta, "knit_meta_id"): 'xfun::attr()' está en desuso.
## Utilizar 'xfun::attr2()' en su lugar.
## Ver help("Deprecated")
## Warning in attr(.knitEnv$meta, "knit_meta_id"): 'xfun::attr()' está en desuso.
## Utilizar 'xfun::attr2()' en su lugar.
## Ver help("Deprecated")
## Warning in attr(.knitEnv$meta, "knit_meta_id"): 'xfun::attr()' está en desuso.
## Utilizar 'xfun::attr2()' en su lugar.
## Ver help("Deprecated")
## Warning in attr(.knitEnv$meta, "knit_meta_id"): 'xfun::attr()' está en desuso.
## Utilizar 'xfun::attr2()' en su lugar.
## Ver help("Deprecated")
## Warning in attr(.knitEnv$meta, "knit_meta_id"): 'xfun::attr()' está en desuso.
## Utilizar 'xfun::attr2()' en su lugar.
## Ver help("Deprecated")
## Warning in attr(.knitEnv$meta, "knit_meta_id"): 'xfun::attr()' está en desuso.
## Utilizar 'xfun::attr2()' en su lugar.
## Ver help("Deprecated")
\end{verbatim}

\begin{verbatim}
## Warning in attr(x, "align"): 'xfun::attr()' está en desuso.
## Utilizar 'xfun::attr2()' en su lugar.
## Ver help("Deprecated")
\end{verbatim}

\begin{verbatim}
## Warning in attr(x, "format"): 'xfun::attr()' está en desuso.
## Utilizar 'xfun::attr2()' en su lugar.
## Ver help("Deprecated")
\end{verbatim}

\begin{longtable}[t]{ccc}
\caption{\label{tab:unnamed-chunk-20}Resumen: media y moda de pi – Películas 2000}\\
\toprule
Modelo & Media & Moda\\
\midrule
Priori & 0.355 & 0.345\\
Posterior & 0.426 & 0.424\\
\bottomrule
\end{longtable}

La priori refleja la información acumulada de años anteriores, mientras
que la posterior combina esa información con los datos más recientes de
los 2000 , donde \(\pi \sim \text{Beta}(40, 54)\). Como resultado, la
media de \(\pi\) aumenta a 0.43 y la moda a 0.42, indicando un efecto
positivo consistente con la tendencia de películas más modernas
mostrando una mayor proporción de aprobaciones en el test de Bechdel.

\subsubsection{4. Jenna, en cambio, comienza su análisis también con una
priori Beta(1, 1), pero analiza películas de 1980, 1990 y 2000 todas en
el mismo
día.}\label{jenna-en-cambio-comienza-su-anuxe1lisis-tambiuxe9n-con-una-priori-beta1-1-pero-analiza-peluxedculas-de-1980-1990-y-2000-todas-en-el-mismo-duxeda.}

Esta vez analizamos todas las películas de los años anteriores, es
decir, 1980, 1990 y 2000, sumando un total de 92 películas.

\begin{verbatim}
## Warning in attr(x, "align"): 'xfun::attr()' está en desuso.
## Utilizar 'xfun::attr2()' en su lugar.
## Ver help("Deprecated")
\end{verbatim}

\begin{verbatim}
## Warning in attr(.knitEnv$meta, "knit_meta_id"): 'xfun::attr()' está en desuso.
## Utilizar 'xfun::attr2()' en su lugar.
## Ver help("Deprecated")
## Warning in attr(.knitEnv$meta, "knit_meta_id"): 'xfun::attr()' está en desuso.
## Utilizar 'xfun::attr2()' en su lugar.
## Ver help("Deprecated")
## Warning in attr(.knitEnv$meta, "knit_meta_id"): 'xfun::attr()' está en desuso.
## Utilizar 'xfun::attr2()' en su lugar.
## Ver help("Deprecated")
## Warning in attr(.knitEnv$meta, "knit_meta_id"): 'xfun::attr()' está en desuso.
## Utilizar 'xfun::attr2()' en su lugar.
## Ver help("Deprecated")
## Warning in attr(.knitEnv$meta, "knit_meta_id"): 'xfun::attr()' está en desuso.
## Utilizar 'xfun::attr2()' en su lugar.
## Ver help("Deprecated")
## Warning in attr(.knitEnv$meta, "knit_meta_id"): 'xfun::attr()' está en desuso.
## Utilizar 'xfun::attr2()' en su lugar.
## Ver help("Deprecated")
## Warning in attr(.knitEnv$meta, "knit_meta_id"): 'xfun::attr()' está en desuso.
## Utilizar 'xfun::attr2()' en su lugar.
## Ver help("Deprecated")
## Warning in attr(.knitEnv$meta, "knit_meta_id"): 'xfun::attr()' está en desuso.
## Utilizar 'xfun::attr2()' en su lugar.
## Ver help("Deprecated")
## Warning in attr(.knitEnv$meta, "knit_meta_id"): 'xfun::attr()' está en desuso.
## Utilizar 'xfun::attr2()' en su lugar.
## Ver help("Deprecated")
## Warning in attr(.knitEnv$meta, "knit_meta_id"): 'xfun::attr()' está en desuso.
## Utilizar 'xfun::attr2()' en su lugar.
## Ver help("Deprecated")
## Warning in attr(.knitEnv$meta, "knit_meta_id"): 'xfun::attr()' está en desuso.
## Utilizar 'xfun::attr2()' en su lugar.
## Ver help("Deprecated")
## Warning in attr(.knitEnv$meta, "knit_meta_id"): 'xfun::attr()' está en desuso.
## Utilizar 'xfun::attr2()' en su lugar.
## Ver help("Deprecated")
## Warning in attr(.knitEnv$meta, "knit_meta_id"): 'xfun::attr()' está en desuso.
## Utilizar 'xfun::attr2()' en su lugar.
## Ver help("Deprecated")
## Warning in attr(.knitEnv$meta, "knit_meta_id"): 'xfun::attr()' está en desuso.
## Utilizar 'xfun::attr2()' en su lugar.
## Ver help("Deprecated")
\end{verbatim}

\begin{verbatim}
## Warning in attr(x, "align"): 'xfun::attr()' está en desuso.
## Utilizar 'xfun::attr2()' en su lugar.
## Ver help("Deprecated")
\end{verbatim}

\begin{verbatim}
## Warning in attr(x, "format"): 'xfun::attr()' está en desuso.
## Utilizar 'xfun::attr2()' en su lugar.
## Ver help("Deprecated")
\end{verbatim}

\begin{longtable}[t]{ccc}
\caption{\label{tab:unnamed-chunk-23}Películas 1980-2000 – Test de Bechdel}\\
\toprule
Resultado & Cantidad & Proporción\\
\midrule
FAIL & 53 & 0.576\\
PASS & 39 & 0.424\\
Total & 92 & 1.000\\
\bottomrule
\end{longtable}

\pandocbounded{\includegraphics[keepaspectratio]{tarea_5_files/figure-latex/unnamed-chunk-24-1.pdf}}

\begin{verbatim}
## Warning in attr(x, "align"): 'xfun::attr()' está en desuso.
## Utilizar 'xfun::attr2()' en su lugar.
## Ver help("Deprecated")
\end{verbatim}

\begin{verbatim}
## Warning in attr(.knitEnv$meta, "knit_meta_id"): 'xfun::attr()' está en desuso.
## Utilizar 'xfun::attr2()' en su lugar.
## Ver help("Deprecated")
## Warning in attr(.knitEnv$meta, "knit_meta_id"): 'xfun::attr()' está en desuso.
## Utilizar 'xfun::attr2()' en su lugar.
## Ver help("Deprecated")
## Warning in attr(.knitEnv$meta, "knit_meta_id"): 'xfun::attr()' está en desuso.
## Utilizar 'xfun::attr2()' en su lugar.
## Ver help("Deprecated")
## Warning in attr(.knitEnv$meta, "knit_meta_id"): 'xfun::attr()' está en desuso.
## Utilizar 'xfun::attr2()' en su lugar.
## Ver help("Deprecated")
## Warning in attr(.knitEnv$meta, "knit_meta_id"): 'xfun::attr()' está en desuso.
## Utilizar 'xfun::attr2()' en su lugar.
## Ver help("Deprecated")
## Warning in attr(.knitEnv$meta, "knit_meta_id"): 'xfun::attr()' está en desuso.
## Utilizar 'xfun::attr2()' en su lugar.
## Ver help("Deprecated")
## Warning in attr(.knitEnv$meta, "knit_meta_id"): 'xfun::attr()' está en desuso.
## Utilizar 'xfun::attr2()' en su lugar.
## Ver help("Deprecated")
## Warning in attr(.knitEnv$meta, "knit_meta_id"): 'xfun::attr()' está en desuso.
## Utilizar 'xfun::attr2()' en su lugar.
## Ver help("Deprecated")
## Warning in attr(.knitEnv$meta, "knit_meta_id"): 'xfun::attr()' está en desuso.
## Utilizar 'xfun::attr2()' en su lugar.
## Ver help("Deprecated")
## Warning in attr(.knitEnv$meta, "knit_meta_id"): 'xfun::attr()' está en desuso.
## Utilizar 'xfun::attr2()' en su lugar.
## Ver help("Deprecated")
## Warning in attr(.knitEnv$meta, "knit_meta_id"): 'xfun::attr()' está en desuso.
## Utilizar 'xfun::attr2()' en su lugar.
## Ver help("Deprecated")
## Warning in attr(.knitEnv$meta, "knit_meta_id"): 'xfun::attr()' está en desuso.
## Utilizar 'xfun::attr2()' en su lugar.
## Ver help("Deprecated")
## Warning in attr(.knitEnv$meta, "knit_meta_id"): 'xfun::attr()' está en desuso.
## Utilizar 'xfun::attr2()' en su lugar.
## Ver help("Deprecated")
## Warning in attr(.knitEnv$meta, "knit_meta_id"): 'xfun::attr()' está en desuso.
## Utilizar 'xfun::attr2()' en su lugar.
## Ver help("Deprecated")
\end{verbatim}

\begin{verbatim}
## Warning in attr(x, "align"): 'xfun::attr()' está en desuso.
## Utilizar 'xfun::attr2()' en su lugar.
## Ver help("Deprecated")
\end{verbatim}

\begin{verbatim}
## Warning in attr(x, "format"): 'xfun::attr()' está en desuso.
## Utilizar 'xfun::attr2()' en su lugar.
## Ver help("Deprecated")
\end{verbatim}

\begin{longtable}[t]{ccc}
\caption{\label{tab:unnamed-chunk-26}Resumen: media y moda de pi – Películas 1980-2000}\\
\toprule
Modelo & Media & Moda\\
\midrule
Priori & 0.500 & NA\\
Posterior & 0.426 & 0.424\\
\bottomrule
\end{longtable}

Al analizar juntas las películas de 1980, 1990 y 2000, Jenna parte de
una priori no informativa (Beta(1,1)) y obtiene una posterior con media
de 0.43 y moda de 0.42 que distribuye \(\pi \sim \text{Beta}(40, 54)\),
reflejando la evidencia combinada de todas las décadas. Comparando con
John, que fue actualizando su posterior día a día, vemos que ambos
enfoques llevan a la misma distribución y la misma conclusión: la
proporción de películas que pasan el test de Bechdel aumenta con el
tiempo. Sin embargo, mientras la actualización secuencial de John
refleja paso a paso el efecto de cada año, el enfoque de Jenna combina
toda la información de una sola vez, produciendo directamente una
estimación consolidada de la tendencia general a lo largo de las tres
décadas.

\newpage

\subsection{Ejercicio 5.7 (Copa del
Mundo)}\label{ejercicio-5.7-copa-del-mundo}

Sea \(\lambda\) el número promedio de goles anotados en un partido de la
Copa Mundial Femenina. Analizaremos \(\lambda\) mediante el siguiente
modelo Gamma-Poisson, donde los datos \(Y_i\) son el número observado de
goles en una muestra de partidos de la Copa del Mundo:

\[
Y_i \mid \lambda \sim \text{Pois}(\lambda), \quad
\lambda \sim \text{Gamma}(1, 0.25)
\]

\subsubsection{\texorpdfstring{1. Grafique y resuma nuestro conocimiento
priori sobre
\(\lambda\)}{1. Grafique y resuma nuestro conocimiento priori sobre \textbackslash lambda}}\label{grafique-y-resuma-nuestro-conocimiento-priori-sobre-lambda}

\pandocbounded{\includegraphics[keepaspectratio]{tarea_5_files/figure-latex/unnamed-chunk-27-1.pdf}}

\begin{verbatim}
## Warning in attr(x, "align"): 'xfun::attr()' está en desuso.
## Utilizar 'xfun::attr2()' en su lugar.
## Ver help("Deprecated")
\end{verbatim}

\begin{verbatim}
## Warning in attr(.knitEnv$meta, "knit_meta_id"): 'xfun::attr()' está en desuso.
## Utilizar 'xfun::attr2()' en su lugar.
## Ver help("Deprecated")
## Warning in attr(.knitEnv$meta, "knit_meta_id"): 'xfun::attr()' está en desuso.
## Utilizar 'xfun::attr2()' en su lugar.
## Ver help("Deprecated")
## Warning in attr(.knitEnv$meta, "knit_meta_id"): 'xfun::attr()' está en desuso.
## Utilizar 'xfun::attr2()' en su lugar.
## Ver help("Deprecated")
## Warning in attr(.knitEnv$meta, "knit_meta_id"): 'xfun::attr()' está en desuso.
## Utilizar 'xfun::attr2()' en su lugar.
## Ver help("Deprecated")
## Warning in attr(.knitEnv$meta, "knit_meta_id"): 'xfun::attr()' está en desuso.
## Utilizar 'xfun::attr2()' en su lugar.
## Ver help("Deprecated")
## Warning in attr(.knitEnv$meta, "knit_meta_id"): 'xfun::attr()' está en desuso.
## Utilizar 'xfun::attr2()' en su lugar.
## Ver help("Deprecated")
## Warning in attr(.knitEnv$meta, "knit_meta_id"): 'xfun::attr()' está en desuso.
## Utilizar 'xfun::attr2()' en su lugar.
## Ver help("Deprecated")
## Warning in attr(.knitEnv$meta, "knit_meta_id"): 'xfun::attr()' está en desuso.
## Utilizar 'xfun::attr2()' en su lugar.
## Ver help("Deprecated")
## Warning in attr(.knitEnv$meta, "knit_meta_id"): 'xfun::attr()' está en desuso.
## Utilizar 'xfun::attr2()' en su lugar.
## Ver help("Deprecated")
## Warning in attr(.knitEnv$meta, "knit_meta_id"): 'xfun::attr()' está en desuso.
## Utilizar 'xfun::attr2()' en su lugar.
## Ver help("Deprecated")
## Warning in attr(.knitEnv$meta, "knit_meta_id"): 'xfun::attr()' está en desuso.
## Utilizar 'xfun::attr2()' en su lugar.
## Ver help("Deprecated")
## Warning in attr(.knitEnv$meta, "knit_meta_id"): 'xfun::attr()' está en desuso.
## Utilizar 'xfun::attr2()' en su lugar.
## Ver help("Deprecated")
## Warning in attr(.knitEnv$meta, "knit_meta_id"): 'xfun::attr()' está en desuso.
## Utilizar 'xfun::attr2()' en su lugar.
## Ver help("Deprecated")
## Warning in attr(.knitEnv$meta, "knit_meta_id"): 'xfun::attr()' está en desuso.
## Utilizar 'xfun::attr2()' en su lugar.
## Ver help("Deprecated")
\end{verbatim}

\begin{verbatim}
## Warning in attr(x, "align"): 'xfun::attr()' está en desuso.
## Utilizar 'xfun::attr2()' en su lugar.
## Ver help("Deprecated")
\end{verbatim}

\begin{verbatim}
## Warning in attr(x, "format"): 'xfun::attr()' está en desuso.
## Utilizar 'xfun::attr2()' en su lugar.
## Ver help("Deprecated")
\end{verbatim}

\begin{longtable}[t]{cccc}
\caption{\label{tab:unnamed-chunk-29}Resumen de la distribucion prior Gamma(1, 0.25)}\\
\toprule
Media & Moda & Varianza & Desviacion\\
\midrule
4 & 0 & 16 & 4\\
\bottomrule
\end{longtable}

La distribución a priori de \(\lambda\) es asimétrica y sesgada hacia la
derecha, reflejando que creemos que valores pequeños de goles por
partido son más probables, aunque no descartamos valores mayores. La
media de la prior es 4, la varianza es 16, lo que indica bastante
incertidumbre sobre \(\lambda\), y la probabilidad más alta a priori se
concentra cerca de valores bajos.

\subsubsection{\texorpdfstring{2. ¿Por qué es razonable usar un modelo
Poisson para nuestros datos
\(Y_i\)?}{2. ¿Por qué es razonable usar un modelo Poisson para nuestros datos Y\_i?}}\label{por-quuxe9-es-razonable-usar-un-modelo-poisson-para-nuestros-datos-y_i}

Cada \(Y_i\) representa el número de goles anotados en un partido, es
decir, un conteo de eventos discretos que ocurren de manera
independiente durante el encuentro. La distribución Poisson es adecuada
porque modela eventos que suceden a una tasa promedio constante
(\(\lambda\)) dentro de un intervalo fijo, en este caso, la duración del
partido, y además no requiere un límite máximo de goles, ya que puede
tomar cualquier valor entero no negativo.

\subsubsection{\texorpdfstring{3. El conjunto de datos
\texttt{wwc\_2019\_matches} del paquete \texttt{fivethirtyeight} incluye
el número de goles anotados por los dos equipos en cada partido de la
Copa Mundial Femenina 2019. Defina, grafique y discuta el número total
de goles por
partido:}{3. El conjunto de datos wwc\_2019\_matches del paquete fivethirtyeight incluye el número de goles anotados por los dos equipos en cada partido de la Copa Mundial Femenina 2019. Defina, grafique y discuta el número total de goles por partido:}}\label{el-conjunto-de-datos-wwc_2019_matches-del-paquete-fivethirtyeight-incluye-el-nuxfamero-de-goles-anotados-por-los-dos-equipos-en-cada-partido-de-la-copa-mundial-femenina-2019.-defina-grafique-y-discuta-el-nuxfamero-total-de-goles-por-partido}

\begin{Shaded}
\begin{Highlighting}[]
\FunctionTok{library}\NormalTok{(fivethirtyeight)}
\FunctionTok{data}\NormalTok{(}\StringTok{"wwc\_2019\_matches"}\NormalTok{)}
\NormalTok{wwc\_2019\_matches }\OtherTok{\textless{}{-}}\NormalTok{ wwc\_2019\_matches }\SpecialCharTok{\%\textgreater{}\%} 
  \FunctionTok{mutate}\NormalTok{(}\AttributeTok{total\_goals =}\NormalTok{ score1 }\SpecialCharTok{+}\NormalTok{ score2)}
\end{Highlighting}
\end{Shaded}

\pandocbounded{\includegraphics[keepaspectratio]{tarea_5_files/figure-latex/unnamed-chunk-31-1.pdf}}

\begin{verbatim}
## Warning in attr(x, "align"): 'xfun::attr()' está en desuso.
## Utilizar 'xfun::attr2()' en su lugar.
## Ver help("Deprecated")
\end{verbatim}

\begin{verbatim}
## Warning in attr(.knitEnv$meta, "knit_meta_id"): 'xfun::attr()' está en desuso.
## Utilizar 'xfun::attr2()' en su lugar.
## Ver help("Deprecated")
## Warning in attr(.knitEnv$meta, "knit_meta_id"): 'xfun::attr()' está en desuso.
## Utilizar 'xfun::attr2()' en su lugar.
## Ver help("Deprecated")
## Warning in attr(.knitEnv$meta, "knit_meta_id"): 'xfun::attr()' está en desuso.
## Utilizar 'xfun::attr2()' en su lugar.
## Ver help("Deprecated")
## Warning in attr(.knitEnv$meta, "knit_meta_id"): 'xfun::attr()' está en desuso.
## Utilizar 'xfun::attr2()' en su lugar.
## Ver help("Deprecated")
## Warning in attr(.knitEnv$meta, "knit_meta_id"): 'xfun::attr()' está en desuso.
## Utilizar 'xfun::attr2()' en su lugar.
## Ver help("Deprecated")
## Warning in attr(.knitEnv$meta, "knit_meta_id"): 'xfun::attr()' está en desuso.
## Utilizar 'xfun::attr2()' en su lugar.
## Ver help("Deprecated")
## Warning in attr(.knitEnv$meta, "knit_meta_id"): 'xfun::attr()' está en desuso.
## Utilizar 'xfun::attr2()' en su lugar.
## Ver help("Deprecated")
## Warning in attr(.knitEnv$meta, "knit_meta_id"): 'xfun::attr()' está en desuso.
## Utilizar 'xfun::attr2()' en su lugar.
## Ver help("Deprecated")
## Warning in attr(.knitEnv$meta, "knit_meta_id"): 'xfun::attr()' está en desuso.
## Utilizar 'xfun::attr2()' en su lugar.
## Ver help("Deprecated")
## Warning in attr(.knitEnv$meta, "knit_meta_id"): 'xfun::attr()' está en desuso.
## Utilizar 'xfun::attr2()' en su lugar.
## Ver help("Deprecated")
## Warning in attr(.knitEnv$meta, "knit_meta_id"): 'xfun::attr()' está en desuso.
## Utilizar 'xfun::attr2()' en su lugar.
## Ver help("Deprecated")
## Warning in attr(.knitEnv$meta, "knit_meta_id"): 'xfun::attr()' está en desuso.
## Utilizar 'xfun::attr2()' en su lugar.
## Ver help("Deprecated")
## Warning in attr(.knitEnv$meta, "knit_meta_id"): 'xfun::attr()' está en desuso.
## Utilizar 'xfun::attr2()' en su lugar.
## Ver help("Deprecated")
## Warning in attr(.knitEnv$meta, "knit_meta_id"): 'xfun::attr()' está en desuso.
## Utilizar 'xfun::attr2()' en su lugar.
## Ver help("Deprecated")
\end{verbatim}

\begin{verbatim}
## Warning in attr(x, "align"): 'xfun::attr()' está en desuso.
## Utilizar 'xfun::attr2()' en su lugar.
## Ver help("Deprecated")
\end{verbatim}

\begin{verbatim}
## Warning in attr(x, "format"): 'xfun::attr()' está en desuso.
## Utilizar 'xfun::attr2()' en su lugar.
## Ver help("Deprecated")
\end{verbatim}

\begin{longtable}[t]{cc}
\caption{\label{tab:unnamed-chunk-31}Resumen del total de goles por partido – Copa Mundial Femenina 2019}\\
\toprule
Estadistico & Total\_Goles\\
\midrule
Minimo & 0.00\\
1er Cuartil & 2.00\\
Mediana & 3.00\\
Media & 2.81\\
3er Cuartil & 3.00\\
\addlinespace
Maximo & 13.00\\
\bottomrule
\end{longtable}

Al sumar los goles de ambos equipos, definimos el total de goles por
partido (total\_goals). La distribución de estos totales, visualizada
mediante un histograma, muestra que la mayoría de los partidos tuvieron
entre 2 y 5 goles, con una mediana de 3 y un promedio de aproximadamente
2.81 goles por partido. Aunque hay algunos partidos con valores extremos
(hasta 13 goles), la distribución es ligeramente sesgada a la derecha,
indicando que los partidos con muchos goles son menos frecuentes. Esta
información sugiere que modelar los goles por partido con un enfoque de
conteo discreto, como la distribución Poisson y el \(\lambda\) que
elegimos, es razonable.

\newpage

\subsubsection{\texorpdfstring{4. Identifique el modelo posterior de
\(\lambda\) y verifique su respuesta usando
\texttt{summarize\_gamma\_poisson()}.}{4. Identifique el modelo posterior de \textbackslash lambda y verifique su respuesta usando summarize\_gamma\_poisson().}}\label{identifique-el-modelo-posterior-de-lambda-y-verifique-su-respuesta-usando-summarize_gamma_poisson.}

\begin{verbatim}
## Warning in attr(x, "align"): 'xfun::attr()' está en desuso.
## Utilizar 'xfun::attr2()' en su lugar.
## Ver help("Deprecated")
\end{verbatim}

\begin{verbatim}
## Warning in attr(.knitEnv$meta, "knit_meta_id"): 'xfun::attr()' está en desuso.
## Utilizar 'xfun::attr2()' en su lugar.
## Ver help("Deprecated")
## Warning in attr(.knitEnv$meta, "knit_meta_id"): 'xfun::attr()' está en desuso.
## Utilizar 'xfun::attr2()' en su lugar.
## Ver help("Deprecated")
## Warning in attr(.knitEnv$meta, "knit_meta_id"): 'xfun::attr()' está en desuso.
## Utilizar 'xfun::attr2()' en su lugar.
## Ver help("Deprecated")
## Warning in attr(.knitEnv$meta, "knit_meta_id"): 'xfun::attr()' está en desuso.
## Utilizar 'xfun::attr2()' en su lugar.
## Ver help("Deprecated")
## Warning in attr(.knitEnv$meta, "knit_meta_id"): 'xfun::attr()' está en desuso.
## Utilizar 'xfun::attr2()' en su lugar.
## Ver help("Deprecated")
## Warning in attr(.knitEnv$meta, "knit_meta_id"): 'xfun::attr()' está en desuso.
## Utilizar 'xfun::attr2()' en su lugar.
## Ver help("Deprecated")
## Warning in attr(.knitEnv$meta, "knit_meta_id"): 'xfun::attr()' está en desuso.
## Utilizar 'xfun::attr2()' en su lugar.
## Ver help("Deprecated")
## Warning in attr(.knitEnv$meta, "knit_meta_id"): 'xfun::attr()' está en desuso.
## Utilizar 'xfun::attr2()' en su lugar.
## Ver help("Deprecated")
## Warning in attr(.knitEnv$meta, "knit_meta_id"): 'xfun::attr()' está en desuso.
## Utilizar 'xfun::attr2()' en su lugar.
## Ver help("Deprecated")
## Warning in attr(.knitEnv$meta, "knit_meta_id"): 'xfun::attr()' está en desuso.
## Utilizar 'xfun::attr2()' en su lugar.
## Ver help("Deprecated")
## Warning in attr(.knitEnv$meta, "knit_meta_id"): 'xfun::attr()' está en desuso.
## Utilizar 'xfun::attr2()' en su lugar.
## Ver help("Deprecated")
## Warning in attr(.knitEnv$meta, "knit_meta_id"): 'xfun::attr()' está en desuso.
## Utilizar 'xfun::attr2()' en su lugar.
## Ver help("Deprecated")
## Warning in attr(.knitEnv$meta, "knit_meta_id"): 'xfun::attr()' está en desuso.
## Utilizar 'xfun::attr2()' en su lugar.
## Ver help("Deprecated")
## Warning in attr(.knitEnv$meta, "knit_meta_id"): 'xfun::attr()' está en desuso.
## Utilizar 'xfun::attr2()' en su lugar.
## Ver help("Deprecated")
\end{verbatim}

\begin{verbatim}
## Warning in attr(x, "align"): 'xfun::attr()' está en desuso.
## Utilizar 'xfun::attr2()' en su lugar.
## Ver help("Deprecated")
\end{verbatim}

\begin{verbatim}
## Warning in attr(x, "format"): 'xfun::attr()' está en desuso.
## Utilizar 'xfun::attr2()' en su lugar.
## Ver help("Deprecated")
\end{verbatim}

\begin{longtable}[t]{ccccccc}
\caption{\label{tab:unnamed-chunk-33}Resumen de la distribución prior y posterior de lambda}\\
\toprule
Modelo & Shape & Rate & Media & Moda & Varianza & Desviacion\\
\midrule
Priori & 1 & 0.25 & 4.000 & 0.000 & 16.000 & 4.000\\
Posterior & 147 & 52.25 & 2.813 & 2.794 & 0.054 & 0.232\\
\bottomrule
\end{longtable}

\subsubsection{\texorpdfstring{5. Grafique la función de densidad
priori, la función de verosimilitud (likelihood) y la densidad posterior
de \(\lambda\). Describa cómo evoluciona nuestro conocimiento de
\(\lambda\) desde la prior hasta la
posterior.}{5. Grafique la función de densidad priori, la función de verosimilitud (likelihood) y la densidad posterior de \textbackslash lambda. Describa cómo evoluciona nuestro conocimiento de \textbackslash lambda desde la prior hasta la posterior.}}\label{grafique-la-funciuxf3n-de-densidad-priori-la-funciuxf3n-de-verosimilitud-likelihood-y-la-densidad-posterior-de-lambda.-describa-cuxf3mo-evoluciona-nuestro-conocimiento-de-lambda-desde-la-prior-hasta-la-posterior.}

\pandocbounded{\includegraphics[keepaspectratio]{tarea_5_files/figure-latex/unnamed-chunk-34-1.pdf}}
El gráfico muestra la evolución de nuestra creencia sobre \(\lambda\),
el promedio de goles por partido. La prior Gamma(1, 0.25) es fuertemente
sesgada hacia la derecha, reflejando que inicialmente creíamos que
valores pequeños de \(\lambda\) eran más probables, aunque permitíamos
valores grandes. En la posterior la varianza disminuye notablemente. En
conjunto con la likelihood, esto muestra cómo los datos recientes
ajustan y reducen la incertidumbre de nuestra estimación de \(\lambda\).

\section{Ejercicio 5.12 (Cerebros de
control)}\label{ejercicio-5.12-cerebros-de-control}

Sea \(\mu\) el \textbf{volumen medio} del hipocampo entre personas que
\textbf{no han sido diagnosticadas con una conmoción}.\\
Analizaremos \(\mu\) con el siguiente \textbf{modelo Normal--Normal},
donde los datos \(Y\) representan los volúmenes hipocampales de los
individuos de este grupo:

\[
Y \mid \mu \stackrel{ind}{\sim} \text{Normal}(\mu,\, \sigma = 0.5), 
\qquad 
\mu \sim \text{Normal}(\theta = 6.5,\, \tau = 0.4).
\]

\begin{center}\rule{0.5\linewidth}{0.5pt}\end{center}

\subsection{Tareas}\label{tareas}

\begin{enumerate}
\def\labelenumi{\arabic{enumi}.}
\tightlist
\item
  Usar los datos \texttt{football} para calcular la \textbf{media
  muestral} del volumen hipocampal y el \textbf{tamaño muestral} de los
  sujetos de control que no han sido diagnosticados con una conmoción.
\end{enumerate}

El grupo de control está compuesto por \(n = 25\) sujetos, con un
volumen hipocampal medio de \(\bar{y} = 7.6\) centímetros cúbicos.

\begin{enumerate}
\def\labelenumi{\arabic{enumi}.}
\setcounter{enumi}{1}
\tightlist
\item
  Identificar el \textbf{modelo posterior} de \(\mu\) y verificar la
  respuesta con \texttt{summarize\_normal\_normal()}. \#\# Paso 2:
  Modelo posterior
\end{enumerate}

La distribución posterior de \(\mu\) bajo el modelo Normal--Normal es:

\[
\mu \mid y \sim \text{Normal}\!\Bigg(
\frac{\tfrac{\theta}{\tau^2} + \tfrac{n \bar{y}}{\sigma^2}}
     {\tfrac{1}{\tau^2} + \tfrac{n}{\sigma^2}},
\quad
\frac{1}{\tfrac{1}{\tau^2} + \tfrac{n}{\sigma^2}}
\Bigg)
\]

donde \(\bar{y} = 7.6\), \(n = 25\), \(\sigma = 0.5\), \(\theta = 6.5\)
y \(\tau = 0.4\).

\begin{Shaded}
\begin{Highlighting}[]
\FunctionTok{summarize\_normal\_normal}\NormalTok{(}\AttributeTok{mean =} \FloatTok{6.5}\NormalTok{, }\AttributeTok{sd =} \FloatTok{0.4}\NormalTok{, }\AttributeTok{sigma =} \FloatTok{0.5}\NormalTok{,}
                        \AttributeTok{y\_bar =} \FloatTok{7.6}\NormalTok{, }\AttributeTok{n =} \DecValTok{25}\NormalTok{)}
\end{Highlighting}
\end{Shaded}

\begin{verbatim}
##       model     mean     mode         var         sd
## 1     prior 6.500000 6.500000 0.160000000 0.40000000
## 2 posterior 7.535294 7.535294 0.009411765 0.09701425
\end{verbatim}

\begin{enumerate}
\def\labelenumi{\arabic{enumi}.}
\setcounter{enumi}{2}
\tightlist
\item
  Graficar la \textbf{pdf previa}, la \textbf{función de verosimilitud}
  y la \textbf{pdf posterior} de \(\mu\).
  \pandocbounded{\includegraphics[keepaspectratio]{tarea_5_files/figure-latex/unnamed-chunk-38-1.pdf}}
\end{enumerate}

\pandocbounded{\includegraphics[keepaspectratio]{tarea_5_files/figure-latex/unnamed-chunk-39-1.pdf}}

\begin{enumerate}
\def\labelenumi{\arabic{enumi}.}
\setcounter{enumi}{3}
\tightlist
\item
  Describir cómo evoluciona la comprensión de \(\mu\) desde la previa
  hasta la posterior.
\end{enumerate}

\subsection{\texorpdfstring{Paso 4: Evolución de la comprensión de
\(\mu\)}{Paso 4: Evolución de la comprensión de \textbackslash mu}}\label{paso-4-evoluciuxf3n-de-la-comprensiuxf3n-de-mu}

En la \textbf{distribución previa}, nuestro conocimiento sobre \(\mu\)
estaba centrado en \(\theta = 6.5\) con bastante dispersión
(\(\tau = 0.4\)).\\
La curva amarilla refleja esta creencia inicial, amplia e incierta.

La \textbf{verosimilitud} (curva azul) incorpora la información de los
datos: la media muestral observada en el grupo de control fue
\(\bar{y} = 7.6\) con \(n = 25\) y \(\sigma = 0.5\).\\
Esta curva está mucho más concentrada y centrada cerca de 7.6, lo que
muestra que los datos son bastante informativos.

La \textbf{distribución posterior} (curva verde) combina la previa con
los datos.\\
Se observa cómo se desplaza hacia valores cercanos a 7.5 y se concentra
fuertemente en torno a ese punto
(\(\text{sd posterior} \approx 0.097\)).

En conclusión, los datos actualizan de forma sustancial nuestra
comprensión:\\
pasamos de una creencia amplia y centrada en 6.5 a una estimación más
precisa, muy cercana a la media observada en la muestra de control.

Ejercicio Normal-Normal (guiado con el libro):

\begin{enumerate}
\def\labelenumi{(\roman{enumi})}
\item
  Calcular la distribución a posteriori para \(μ\).
\item
  Demostrar que la distribución a posteriori para μ es normal con media
  igual a un promedio ponderado de la media a priori \(θ\) y de la media
  muestral.
\item
  Analizar la influencia del tamaño muestral \(n\) sobre la media y la
  varianza de la distribución a posterior para \(μ\). Conclusión: la
  distribución a priori normal para \(μ\) es conjugada con el modelo
  muestral de observaciones independientes de una población normal con
  media \(μ\) y varianza conocida \(σ^2\)
\end{enumerate}

\section{Ejercicio Normal--Normal (guiado con el
libro)}\label{ejercicio-normalnormal-guiado-con-el-libro}

Sea \(\mu\) el volumen medio del hipocampo. Supondremos el
\textbf{modelo Normal--Normal} con varianza conocida \(\sigma^2\):

\[
Y_i \mid \mu \stackrel{ind}{\sim} \mathcal{N}(\mu,\sigma^2), \quad i=1,\dots,n,
\qquad
\mu \sim \mathcal{N}(\theta,\tau^2).
\]

Denotemos \(\bar{y}=\frac{1}{n}\sum_{i=1}^n y_i\).

\begin{center}\rule{0.5\linewidth}{0.5pt}\end{center}

\subsection{\texorpdfstring{(i) Distribución a posteriori de
\(\mu\)}{(i) Distribución a posteriori de \textbackslash mu}}\label{i-distribuciuxf3n-a-posteriori-de-mu}

La distribución a posteriori es también Normal:

\[
\mu \mid \mathbf{y} \sim \mathcal{N}\!\Big(\, m_n,\; v_n \Big),
\] con \[
v_n \;=\; \frac{1}{\tfrac{1}{\tau^2}+\tfrac{n}{\sigma^2}},
\qquad
m_n \;=\; v_n\!\left(\frac{\theta}{\tau^2}+\frac{n\,\bar{y}}{\sigma^2}\right).
\]

\begin{center}\rule{0.5\linewidth}{0.5pt}\end{center}

\subsection{\texorpdfstring{(ii) Media posterior como \textbf{promedio
ponderado} de \(\theta\) y
\(\bar{y}\)}{(ii) Media posterior como promedio ponderado de \textbackslash theta y \textbackslash bar\{y\}}}\label{ii-media-posterior-como-promedio-ponderado-de-theta-y-bary}

Escribiendo los pesos explícitamente:

\[
m_n \;=\; 
\underbrace{\left(\frac{1/\tau^2}{\,1/\tau^2+n/\sigma^2\,}\right)}_{\text{peso previo}}
\theta
\;+\;
\underbrace{\left(\frac{(n/\sigma^2)}{\,1/\tau^2+n/\sigma^2\,}\right)}_{\text{peso de los datos}}
\bar{y}.
\]

Equivalente, en términos de \textbf{precisiones} (inversas de
varianzas):

\[
m_n \;=\; 
\frac{\text{precisión previa}\cdot \theta + \text{precisión de la media muestral}\cdot \bar{y}}
{\text{precisión previa}+\text{precisión de la media muestral}},
\quad
\text{donde }
\text{precisión previa}=1/\tau^2,\;\;
\text{precisión de la media muestral}=n/\sigma^2.
\]

\begin{center}\rule{0.5\linewidth}{0.5pt}\end{center}

\subsection{\texorpdfstring{(iii) Influencia de \(n\) en la media y
varianza a
posteriori}{(iii) Influencia de n en la media y varianza a posteriori}}\label{iii-influencia-de-n-en-la-media-y-varianza-a-posteriori}

\begin{itemize}
\item
  \textbf{Varianza posterior}: \[
  v_n=\frac{1}{1/\tau^2+n/\sigma^2}
  \quad\Rightarrow\quad
  v_n \downarrow \text{ al crecer } n.
  \] Más datos \(\Rightarrow\) \textbf{más precisión} (menor
  incertidumbre).
\item
  \textbf{Media posterior}: \[
  m_n
  = \Bigg(\frac{1/\tau^2}{1/\tau^2+n/\sigma^2}\Bigg)\theta
  + \Bigg(\frac{n/\sigma^2}{1/\tau^2+n/\sigma^2}\Bigg)\bar{y}.
  \] Al aumentar \(n\), el \textbf{peso de los datos}
  \(\frac{n/\sigma^2}{1/\tau^2+n/\sigma^2}\) crece y \(m_n\) se acerca a
  \(\bar{y}\).

  \begin{itemize}
  \tightlist
  \item
    Límite \(n\to\infty\): \(m_n \to \bar{y}\),
    \(v_n \to \sigma^2/n \to 0\).
  \item
    Límite \(n\to 0\): \(m_n \to \theta\), \(v_n \to \tau^2\).
  \end{itemize}
\end{itemize}

\begin{center}\rule{0.5\linewidth}{0.5pt}\end{center}

\subsection{Conclusión
(conjugación)}\label{conclusiuxf3n-conjugaciuxf3n}

La \textbf{Normal} es conjugada para la media \(\mu\) bajo un modelo
Normal con \(\sigma^2\) conocida: una previa
\(\mathcal{N}(\theta,\tau^2)\) produce una posterior
\(\mathcal{N}(m_n,v_n)\) con: \[
v_n=\frac{1}{1/\tau^2+n/\sigma^2}, 
\qquad
m_n=\frac{\tfrac{\theta}{\tau^2}+\tfrac{n\bar{y}}{\sigma^2}}{\tfrac{1}{\tau^2}+\tfrac{n}{\sigma^2}}.
\] La media posterior es un \textbf{promedio ponderado} entre la
creencia previa \(\theta\) y la evidencia \(\bar{y}\), con pesos
proporcionales a sus \textbf{precisiones} respectivas.

\end{document}
